%\documentclass[11pt]{article}
\documentclass[conference,onecolumn]{IEEEtran}

\usepackage[T1]{fontenc}
\usepackage[utf8]{inputenc}
\usepackage[left=1in,right=1in,top=1in,bottom=0.8in]{geometry}
\usepackage[bookmarks=false,pdfstartview=FitH,bookmarksopen=true]{hyperref}
\usepackage{url}
% this is the preamble
% put all of the above code in here
%\usepackage{setspace}
\setlength{\parskip}{0.3em}

%\makeatletter% since there's an at-sign (@) in the command name
%\renewcommand{\@maketitle}{%
%  \begin{center}
%    \parskip\baselineskip% skip a line between paragraphs in the title block
%    \parindent=0pt% don't indent paragraphs in the title block
%    {\bf\@title}\par
%    \textbf{\@author}\par
%    \@date% remove the percent sign at the beginning of this line if you want the date printed
%  \end{center}
%}
%\makeatother% resets the meaning of the at-sign (@)

\title{Personal Statement}
\author{\IEEEauthorblockN{Yongsen MA} \\
%\IEEEauthorblockA{mayongsen@gmail.com, http://yongsen.github.com}
}

\begin{document}

\maketitle% prints the title block
%\onehalfspacing

I am Yongsen MA from Shanghai Jiao Tong University and I am applying a PhD program on wireless networking at CS,HKU. My interest in wireless networks was first aroused during my undergraduate studies in Shandong University when I chose the final year research project on performance evaluation of Zigbee networks based on NS2. During my research I built up the simulation model and evaluation scripts written in \verb"tcl" and \verb"awk" respectively. The thesis scored 96/100 in the graduation defence and helps me better understand the wireless protocol and network simulator. Since that time, I have been working on wireless communications which helps me gain practical experience on programming, publishing and participating in research projects.

My graduate research at the Center for Intelligent Wireless Networking and Cooperative Control (\href{http://wicnc.sjtu.edu.cn}{i-WiNC2}, SJTU) reinforced my interest in wireless networking and mobile computing, especially on the link quality measurement and modeling in mobile networks. I developed two performance measurement platforms separately for GSM/GSM-R and 802.11n covering wireless cellular networks and WLANs. The first platform is designed for Um interference monitoring in GSM/GSM-R, which is written in C\# based on Microsoft .NET Compact Framework and tested along Beijing-Shanghai high-speed railway. Another platform is the link quality measurement software for MIMO-OFDM WLANs of 802.11n, which is written in Linux C based on Atheros wireless driver \texttt{ath9k} and Linux wireless extension. The software development process has been an integral part of my graduate research and helps me to improve relevant skills in programming, open source coding, wireless networking and mobile computing.

Based on these two platforms, I have written four papers, three patents and one report on channel state estimation and link quality measurement. On the physical layer, I written the paper on dynamic estimation of Rician fading channels in GSM-R networks to proceed accurate local mean power estimation in high-speed railway environments. It introduces the dynamic EM estimation algorithm to reduce the measurement overhead and be adaptive to different propagation environments with guaranteed accuracy. This paper has been published in WCSP'12, and the journal vision is under review of Springer Wireless Networks. I have also authored three patents and owned one software copyright of the system and algorithms on this topic. For the link quality measurement, I have written the paper of packet delivery measurement and prediction in mobile 802.11n networks to derive real-time link adaption with certain performance guaranteeing for mobile MIMO-OFDM systems. This paper presents the online packet delivery modeling framework, which incorporates a novel design by exploiting both packet-level and physical-level metrics, along with the diversity property of multi-configuration simultaneously, which overcomes the channel quality capturing problem in static PDR-RSS models. This online framework also strikes a balance between the measurement overhead and accuracy. My history in writing and publishing helps to strengthen my ability in constructing academic questions and solving engineering problems effectively.

In addition, I also participate in research projects including proposals, reports and deliverables. I was responsible for system design and experiments of the key project of Ministry of Railway, and have written a research report and user manual on it. I also participate in discussions of NSFC projects including dynamic spectrum auction in cognitive radio and demand response in smart grid. Besides, I have the opportunity to participate in conference and journal papers reviewing, mainly including IEEE Infocom, IEEE Globecom, and Springer Wireless Networks. As the instructor of PRP (Participation in Research Program) for undergraduate students, I am in charge of directing students on software developing and thesis writing including a paper on dual-antenna based handover scheme for GSM-R network which is presented in WCSP'12 as well. All these positive research experiences equipped me with additional skills to summarize academic problems and co-ordinate opinions efficiently.

Currently, I am working on the energy-efficient rate adaption in 802.11n networks, seeking the suitable trade-off between data rate and packet delivery under the constraints of energy consumption. I hope to carry on the related researches including energy-efficient algorithm and cross-layer protocol in MIMO-OFDM systems in the near future. This has motivated me to continue into a Doctoral program in wireless networking. My enthusiasm and persistence in research, engineering and publishing will blend well with this degree and enable me make valuable contributions in this field.HKU has a good international reputation and is a stimulating environment providing excellent researching facilities for graduate students. A PhD degree from HKU will avail opportunities for me and let me contribute more effectively than I have in the past.

\end{document}
