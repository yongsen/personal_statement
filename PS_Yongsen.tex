%\documentclass[11pt]{article}
\documentclass[journal,onecolumn]{IEEEtran}

\usepackage[T1]{fontenc}
\usepackage[utf8]{inputenc}
\usepackage[left=1in,right=1in,top=1in,bottom=1in]{geometry}
\usepackage[bookmarks=true,pdfstartview=FitH,bookmarksopen=true]{hyperref}
\usepackage{bookmark}
\usepackage{url}
% this is the preamble
% put all of the above code in here
%\usepackage{setspace}
\setlength{\parskip}{0.5em}
%\usepackage[svgnames]{xcolor}% provides colors for text
%\makeatletter% since there's an at-sign (@) in the command name
%\renewcommand{\@maketitle}{%
%  \begin{center}
%    \parskip\baselineskip% skip a line between paragraphs in the title block
%    \parindent=0pt% don't indent paragraphs in the title block
%    \textcolor{black}{\bf\@title}\par
%    \textbf{\@author}\par
%    %\@date% remove the percent sign at the beginning of this line if you want the date printed
%  \end{center}
%}
%\makeatother% resets the meaning of the at-sign (@)

\title{Personal Statement}
\author{Yongsen MA}
%\date{\today}

\begin{document}

\maketitle% prints the title block
%\onehalfspacing
I am Yongsen MA from Shanghai Jiao Tong University and I am applying a PhD program on wireless networking at CSE, HKUST. My interest in wireless networks was first aroused during my undergraduate studies in Shandong University that I chose the final year research project on performance evaluation of Zigbee networks based on NS2. During my research I built up the simulation model and evaluation scripts written in \verb"tcl" and \verb"awk" respectively. The thesis scored 96/100 in the graduation defence and helps me better understand the wireless protocol and network simulator. Since that time, I have been working on wireless communications which helps me gain practical experience on programming, simulations and experiments.


My graduate research at i-WiNC2 reinforced my interest in wireless networking and mobile computing, especially on the link quality measurement and modeling of mobile networks. 

As the instructor of PRP (Participation in Research Program) for undergraduate students, we developed the Um interface monitoring system for GSM/GSM-R networks. The monitor system is written in C\# based on Microsoft .NET Compact Framework and tested along Beijing-Shanghai high-speed railway. On the physical layer channel state estimation, I written the paper on on-line and dynamic estimation of Rician fading channels in GSM-R networks. It introduces the dynamic EM estimation algorithm to reduce the measurement overhead and be adaptive to different propagation environments with guaranteed accuracy. This paper has been published in WCSP'12, and the journal vision is under review of Springer Wireless Networks. I am also the third author of a paper on dual-antenna based handover scheme for GSM-R network, which is presented in WCSP'12 as well. On the performance measurement of GSM-R networks, I have authored three patents and owned one software copyright.

Recently, I developed a performance measurement software under Linux system for mobile 802.11n networks. The software is written in Linux C based on wireless driver \texttt{ath9k} and Linux wireless extension. The detailed information, documents and source code of my Graduate class projects can be found at my website\footnote{Home page: \url{http://yongsen.github.com}, Source code: \url{https://github.com/yongsen}}. For the link quality measurement, I mainly work on the packet delivery measurement and prediction in mobile 802.11n networks. In this work, we present the online PDR-RSS modeling framework, which incorporates a novel design by exploiting both packet-level and physical-level metrics, along with the diversity property of multi-configuration simultaneously. This online framework also strikes a balance between the measurement overhead and accuracy. Through a real world implementation on our testbed, The experimental results indicate that it can achieve throughput gains up to 40\% under different MIMO configurations.

Additionally, I also participate in research projects including proposals, reports and deliverables. I was responsible for the Key Project of Ministry of Railway on performance measurement in GSM-R networks, and I also participant in discussions of NSFC projects including dynamic spectrum auction in cognitive radio and demand response in smart grid. Besides, I have the opportunity to participate in conference and journal papers reviewing, mainly including IEEE Infocom, IEEE Globecom, and Springer Wireless Networks. I have had positive research experiences with my thesis advisor and members of the Center for Intelligent Wireless Networking and Cooperative Control (i-WiNC2)\footnote{Home page: \url{http://wicnc.sjtu.edu.cn}}. This, along with interesting coursework, has motivated me to continue into a Doctoral program in wireless networking. All these experiences equipped me with additional skills to have a better sense of academical and technical problems.

Currently, I am working on the energy-efficient rate adaption in 802.11n networks, seeking the suitable trade-off between data rate and packet delivery under the constraints of energy consumption. I hope to carry on the related researches including MIMO-OFDM systems and mobile applications in the near future. I wish to continue my researches on the development of wireless monitoring and evaluation systems and the implementation of energy-efficient algorithm and cross-layer protocol.

my research experience strong my ability in publishing, and makes me enthusiastic,
\end{document}
