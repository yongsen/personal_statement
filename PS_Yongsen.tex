\documentclass{article}

\usepackage[left=1in,right=1in,top=1in,bottom=1in]{geometry}
\usepackage[bookmarks=true,pdfstartview=FitH,bookmarksopen=true]{hyperref}
\usepackage{bookmark}
\usepackage{url}
% this is the preamble
% put all of the above code in here

\usepackage[svgnames]{xcolor}% provides colors for text
\makeatletter% since there's an at-sign (@) in the command name
\renewcommand{\@maketitle}{%
  \begin{center}
    \parskip\baselineskip% skip a line between paragraphs in the title block
    \parindent=0pt% don't indent paragraphs in the title block
    \textcolor{black}{\bf\@title}\par
    \textbf{\@author}\par
    %\@date% remove the percent sign at the beginning of this line if you want the date printed
  \end{center}
}
\makeatother% resets the meaning of the at-sign (@)

\title{Personal Statement}
\author{Yongsen MA}
\date{\today}

\begin{document}

\maketitle% prints the title block

\section{Reasons I want to do graduate work in this field.}
Statistics is an extremely interesting and useful field as it is a means by which we can further our understanding of the world. In fact, almost every decision we make is based on a (relative) likelihood which we loosely derive from our current understanding of the state of the world.

I have had positive research experiences with my thesis advisor and members of the Center for Intelligent Wireless Networking and Cooperative Control (i-WiN C2)\footnote{\url{http://wicnc.sjtu.edu.cn}}, where I am interning during the Summer of 2010. This, along with interesting coursework, has motivated me to continue into a Doctoral program in wireless networking.

I wish to contribute my talents to the application of statistical analysis and the development of new statistical methods in areas that improve the conditions of life. To fulfill this goal, I must be engaged in basic statistical research including mathematical statistics and probability theory.

\section{My specific interests and experiences in this field.}
The detailed information and documents of my Graduate class projects can be found at my personal home page \footnote{\url{http://yongsen.github.com}}.

Performance Evaluation of Zigbee Networks based on NS2, written in Otcl and Gawk\footnote{Home page: \url{http://yongsen.github.com/ns2_zigbee}, Source code: \url{https://github.com/yongsen/ns2_zigbee}}.


Developed an Um interface monitoring system for GSM/GSM-R networks\footnote{Home page: \url{http://yongsen.github.com/um_monitoring}, Source code: \url{https://github.com/yongsen/um_monitoring}}, deployed on PC/104
platform and Windows XP Embedded system, written in C\# based on Microsoft .NET Compact Framework, tested along Beijing-Shanghai high-speed railway.

Developed a performance measurement application for mobile 802.11n networks\footnote{Home page: \url{http://yongsen.github.com/graded_802.11n}, Source code: \url{https://github.com/yongsen/graded_802.11n}}, deployed on Atheros WiFi devices and Linux system, written in Linux C based on wireless driver ath9k and Linux wireless extension, tested in laboratory and dormitory sceneries.

\section{Special skills or experiences that may relate to an assistantship.}
Participate in research projects including proposals, reports, hardware selection, software design, simulations, experiments and deliverables: NSFC on dynamic spectrum auction in Cognitive Radio and demand response in Smart Grid, and Key Project of Ministry of Railway on performance measurement in GSM-R networks.

Experience in conference and journal papers reviewing, mainly including IEEE Infocom, IEEE Globecom, and Springer Wireless Networks. Instructor of PRP (Participation in Research Program) for undergraduate students.

Vice Director, Science and Technology Division of Graduate Student Union, Responsible for the organization of academic activities and technical exchanges. Volunteer of World Expo 2010, Shanghai and Registration and venue volunteer of China Satellite Navigation Conference 2011, Shanghai

Languages: C\#, C++, XML, HTML; Linux C, Tcl/Otcl, awk/Gawk, Linux shell. Linux Coding: ath9k, Madwifi, mac80211, Linux wireless extension, mobile applications. Software: NS2/NS3,Wireshark, iperf; Mathematic, Matlab, Gnuplot; \LaTeX, Beamer, Visio. Hardware: PC104/PC104+ platforms, Atheros 802.11 wireless devices, GSM/GSM-R devices.
\section{My career plans.}
wireless driver in Linux kernel

mobile applications on Android

\end{document}
